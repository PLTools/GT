\documentclass[10pt]{beamer}

\usepackage[utf8]{inputenc}
\usepackage[english]{babel}
\usepackage[T1]{fontenc}
\usepackage{helvet}
\usepackage{listings}
\usepackage{color}
\usepackage{tcolorbox}

\newcommand{\lstfont}[1]{\color{#1}\scriptsize\ttfamily}
%-------------------------------------------------------
% DEFFINING AND REDEFINING COMMANDS
%-------------------------------------------------------

% \lstdefinelanguage{ocaml}{
% keywords={let, begin, end, in, match, type, and, fun, 
% function, try, with, class, object, method, of, rec, repeat, until,
% while, not, do, done, as, val, inherit, module, sig, @type, struct, 
% if, then, else, open, virtual, new, fresh},
% sensitive=true,
%     language=[ANSI]C++,
%     showstringspaces=false,
%     %backgroundcolor=\color{black!90},
%     basicstyle=\lstfont{white},
%     identifierstyle=\lstfont{white},
%     keywordstyle=\lstfont{magenta!40},
%     numberstyle=\lstfont{white},
%     stringstyle=\lstfont{cyan},
%     commentstyle=\lstfont{yellow!30},
%     emph={
%         cudaMalloc, cudaFree,
%         __global__, __shared__, __device__, __host__,
%         __syncthreads,
%     },
%     emphstyle={\lstfont{green!60!white}},
%     breaklines=true
% }
\lstdefinelanguage{ocaml}{
    showstringspaces=false,
    %backgroundcolor=\color{black!90},
    basicstyle=\lstfont{black},
    identifierstyle=\lstfont{black},
    keywordstyle=\lstfont{orange!40},
    numberstyle=\lstfont{black},
    stringstyle=\lstfont{green},
    commentstyle=\lstfont{darkgreen!30},
    emph={
        cudaMalloc, cudaFree,
        __global__, __shared__, __device__, __host__,
        __syncthreads,
    },
    emphstyle={\lstfont{green!60!white}},
    breaklines=true
}
\lstset{
basicstyle=\small,
identifierstyle=\ttfamily,
keywordstyle=\bfseries,
commentstyle=\scriptsize\rmfamily,
basewidth={0.5em,0.5em},
fontadjust=true,
%escapechar=$,
language=ocaml
}
% \lstset{
%     language=[ANSI]C++,
%     showstringspaces=false,
%     %backgroundcolor=\color{black!90},
%     basicstyle=\lstfont{white},
%     identifierstyle=\lstfont{white},
%     keywordstyle=\lstfont{magenta!40},
%     numberstyle=\lstfont{white},
%     stringstyle=\lstfont{cyan},
%     commentstyle=\lstfont{yellow!30},
%     emph={
%         cudaMalloc, cudaFree,
%         __global__, __shared__, __device__, __host__,
%         __syncthreads,
%     },
%     emphstyle={\lstfont{green!60!white}},
%     breaklines=true
% }
% colored hyperlinks
\newcommand{\chref}[2]{
  \href{#1}{{\usebeamercolor[bg]{Feather}#2}}
}

%-------------------------------------------------------
% INFORMATION IN THE TITLE PAGE
%-------------------------------------------------------

\title[] % [] is optional - is placed on the bottom of the sidebar on every slide
{ % is placed on the title page
      \textbf{Generic transformers}
}

\subtitle[The Feather Beamer Theme]
{
      \textbf{v. 1.0.0}
}

\author[]
{      \\
      {\ttfamily 1@gmail.com}
}

\institute[]
{
      SPbSU
  
  %there must be an empty line above this line - otherwise some unwanted space is added between the university and the country (I do not know why;( )
}

\date{\today}

%-------------------------------------------------------
% THE BODY OF THE PRESENTATION
%-------------------------------------------------------

\begin{document}

%-------------------------------------------------------
% THE TITLEPAGE
%-------------------------------------------------------


% \begin{frame}[fragile]{111}
% \begin{lstlisting}
% __global__
% void foo()
% {}
% 
% foo<<<n,m>>>();
% \end{lstlisting}
% \end{frame}

% % this is the name of the PDF file for the background
\begin{frame}[plain,noframenumbering] % the plain option removes the header from the title page, noframenumbering removes the numbering of this frame only
  \titlepage % call the title page information from above
\end{frame}

\begin{frame}[fragile]{111}
\begin{lstlisting}[ language=caml] 
class ['self] pattern_desc_with_link mut_trfs_here fself = object
  inherit ['self] html_pattern_desc_t_stub mut_trfs_here fself as super
  method! c_Tpat_var () { Ident.name } nameloc =
    let loc_str = Location.show_location nameloc.Asttypes.loc in
    HTML.ul @@
    HTML.seq
      [ HTML.anchor loc_str @@
        HTML.string @@ Printf.sprintf "%S  from %s" name loc_str
      ]
end
\end{lstlisting}

\end{frame}

% \begin{frame}{User Interface}{Loading the Theme and Theme Options}
% 
%   \begin{block}{The Color Theme}
%     Also you can load only the color theme by writing in the preamble of the {\tt tex} file 
%     
%     \vspace{5pt} 
%     
%     \begin{itemize}
%     \item {\tt \textbackslash usecolortheme\{Feather\}}
%     \end{itemize}
%     
%     \vspace{5pt}
%     
%     ...or to change the colors of the various elements in the theme
%     
%     \vspace{5pt} 
%     \begin{itemize}
%     \item Change the bar colors: \\    
%     {\tt \textbackslash setbeamercolor \{Feather\}\{fg=<color>, bg=<color>\}}
%     
%     \vspace{2pt} 
%     
%     \item Change the color of the structural elements: \\    
%     {\tt \textbackslash setbeamercolor\{structure\}\{fg=<color>\}}
%     
%     \vspace{2pt} 
%     
%     \item Change the frame title text color:\\
%     {\tt \textbackslash setbeamercolor\{frametitle\}\{fg=<color>\}}
%     
%     \vspace{2pt} 
%     
%     \item Change the normal text color background:    
%     {\tt \textbackslash setbeamercolor\{normal text\}\{fg=<color>, bg=<color>\}}
%     \end{itemize}
%   \end{block}
% \end{frame}


\end{document}


