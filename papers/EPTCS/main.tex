\documentclass[submission,copyright,creativecommons]{eptcs}
\providecommand{\event}{ML 2018}  % Name of the event you are submitting to
\usepackage{breakurl}             % Not needed if you use pdflatex only.
\usepackage{underscore}           % Only needed if you use pdflatex.

\usepackage{marginnote}
\usepackage{booktabs} 
\usepackage{amssymb}
\usepackage{amsmath}
\usepackage{mathrsfs}
\usepackage{mathtools}
\usepackage{multirow}
\usepackage{listings}
\usepackage{indentfirst}
\usepackage{verbatim}
\usepackage{amsmath, amssymb}
\usepackage{graphicx}
\usepackage{xcolor}
\usepackage{url}
\usepackage{stmaryrd}
\usepackage{xspace}
\usepackage{comment}
\usepackage{wrapfig}
\usepackage[caption=false]{subfig}
\usepackage{placeins}
\usepackage{tabularx}
\usepackage{ragged2e}
\usepackage{soul}
\usepackage{csquotes}
\usepackage{inconsolata}

\lstdefinelanguage{ocaml}{
keywords={@type, function, fun, let, in, match, with, when, class, type,
object, method, of, rec, repeat, until, while, not, do, done, as, val, inherit,
new, module, sig, deriving, datatype, struct, if, then, else, open, private, virtual, include, success, failure,
assert, true, false, end},
sensitive=true,
commentstyle=\small\itshape\ttfamily,
keywordstyle=\ttfamily\bfseries, %\underbar,
identifierstyle=\ttfamily,
basewidth={0.5em,0.5em},
columns=fixed,
fontadjust=true,
literate={->}{{$\to$}}3 {===}{{$\equiv$}}1 {=/=}{{$\not\equiv$}}1 {|>}{{$\triangleright$}}3 {\\/}{{$\vee$}}2 {/\\}{{$\wedge$}}2 {>=}{{$\ge$}}1 {<=}{{$\le$}} 1,
morecomment=[s]{(*}{*)}
}

\lstset{
mathescape=true,
%basicstyle=\small,
identifierstyle=\ttfamily,
keywordstyle=\bfseries,
commentstyle=\scriptsize\rmfamily,
basewidth={0.5em,0.5em},
fontadjust=true,
language=ocaml
}

\newcommand{\cd}[1]{\texttt{#1}}
\newcommand{\inbr}[1]{\left<#1\right>}
%\pagestyle{plain}
%\sloppy

\title{Generic Programming with Combinators and Objects}

\author{Dmitry Kosarev
  \institute{St. Petersburg State University\\
    JetBrains Research \\
    St. Petersburg, Russia}
\email{Dmitrii.Kosarev@protonmail.ch}
\and
Dmitry Boulytchev
\institute{St. Petersburg State University\\
  JetBrains Research \\
  St. Petersburg, Russia}
\email{dboulytchev@math.spbu.ru}
}

\def\titlerunning{Generic Programming with Combinators and Objects}
\def\authorrunning{D.Kosarev, D.Boulytchev}
\begin{document}
\maketitle

\begin{abstract}
  We present a generic programming framework for \textsc{OCaml} which allows to implement extensible
  transformations 
\end{abstract}

\section{Introduction}

Frederic Brooks in his seminal book on software engineering ``The Mythical Man-Month''~\cite{MMM} has characterized the essence of programming with the following words:

\blockquote{``The programmer, like the poet, works only slightly removed from pure thought-stuff. He builds his castles in the air, from air, creating by exertion of the imagination. Few media of
creation are so flexible, so easy to polish and rework, so readily capable of realizing grand conceptual structures. (As we shall see later, this very tractability has its own problems.)''}

Indeed, the virtuality of programs and flexibility of their representation call for structuring; the lack of proper structure easily leads to disastrous consequences
(as it happened to some real-world industrial projects in the past and constantly happens to some student projects in present time). One of commonly used ways to bring a
structure in software are \emph{datatypes}. Datatypes allow to describe the properties of data, what can and cannot be done, and to some extent they prescribe
the semantics to data structures. Being kept in runtime, datatypes make it possible to implement meta-transformations by analyzing types (\emph{introspection})
or even creating new types on the fly (\emph{reflection}).

However, in statically typed languages, as a rule, types are completely erased after the compilation and do not retained in runtime. This has a huge advantage over
dynamic typing as, first, programs do not need to inspect types at runtime anymore and, second, a whole class of bad runtime behaviors~--- type errors~---
is eliminated. The other side of the coin, however, is that now some transformations, which in untyped languages can be implemented ``once and for all'',
can not be typed and have to be re-implemented for each type of interest. One way to overcome this defficiency is to develop a more powerful type system in
which more functions can be typed; as an example we may mention the support for \emph{ad-hoc} polymorphism in \textsc{Haskell} in the forms of type
classes~\cite{TypeClasses} and type families~\cite{TypeFamilies}. However, due to the totality of typechecking and fundamental undecidability results there
will always be some ``good'' functions which cannot be typed. Another approach, \emph{datatype-generic programming}~\cite{DGP}, is aimed at developing techniques for
implementation of practically important families of type-indexed functions using existing language features. For example, types can be encoded in a substrate
language~\cite{Hinze,InstantGenerics,GenericOCaml}, or a part of type information can be saved for runtime~\cite{SYB,SYBOCaml}, or generic functions for a given
type can be generated at compile-time automatically~\cite{Yallop,PPXLib}. The two approaches we mentioned are in fact complementary~--- the more powerful
type system is the more means for datatype-generic programming the language can incorporate natively. For example, parametric polymorphism makes it possible
to natively express many generic functions like length of list of arbitrary elements, etc.

We present a generic programming library \textsc{GT}\footnote{\url{https://github.com/kakadu/GT/tree/ppx}} (\emph{Generic Transformers}), which has been in an
active development and use since 2014. One of the important observations, which motivated the development of our framework, was that many generic functions
can be considered as a modifications of some other generic functions. While our approach is generative~--- we generate generic functioinality from type definitions~---
it also makes possible for end users to easily derive variants of generated functions by redefining some parts of their functionality. This is achieved using
a method-per-constructor encoding for concrete transformations, which resembles the approach of object algebras~\cite{ObjectAlgebras}.

The main properties of our solution are as follows:

\begin{itemize}
\item each transformation is expressed in terms of a \emph{traversal function} and a \emph{transformation object}, which encapsulate the ``interesting part''
  of the transformation;
\item the traversal function is unique for given type and all transformation objects for the type are instances of a unique class;
\item both the traversal function and the class are generated from type definition; we support regular ADTs, structures, polymorphic variants and predefined types;
\item we provide a number of plugins which generate practically important transformations in the form of concrete transformation classes;
\item the plugin system is extensible: end users can implement their own plugins.
\end{itemize}

The library we present is an inheritor of our earlier work~\cite{SYBOCaml} on implementation of ``Scrap Your Boilerplate'' approach~\cite{SYB,SYB1,SYB2}. However,
our experience has shown, that the expressivity and extensibility of SYB is insufficient; in addition the uniform transformations, based solely on type discrimination, turned out to be
inconvenient to use. Our idea initially was to combine combinator and object-oriented approaches~--- the former would provide means for parameterization, while the
latter~--- for extensibility via late binding utilization. This idea in the form of a certain design pattern was successfully evaluated~\cite{SCICO} and then reified
in a library and a syntax extension~\cite{TransformationObjects}. Our follow-up experience with the library~\cite{OCanren} has (once again) shown some flaws in the
implementation. The version we present here is almost a complete re-implementation with these flaws fixed.

The rest of the paper is organized as follows. ...

Acknowledgements ...


\input{exposition.tex}
\begin{figure}[t]
  \center
  \begin{tabular}{c|l}
    \hline
    \multicolumn{2}{c}{\cd{camlp5} version}\\
    \hline
    \lstinline|@type ... $[$ with  $p_1, p_2, \dots$ $]$| & a syntax construct to generate a support for a type \\
                                                         & with plugins $p_1, p_2, \dots$; mutually recursive definitions \\
                                                         & are supported \\
    \lstinline|@$typ$| & the name for a virtual class for type $typ$ \\
    \lstinline|@$plugin$[$typ$]| & the name for a plugin class for type $typ$\\
    \hline
        \multicolumn{2}{c}{\cd{ppxlib} version}\\
    \hline
  \end{tabular}
  \caption{Extended Syntax Constructs}
  \label{syntax}
\end{figure}

\section{Implementation}

The main components of our solution are syntax extensions (both in terms of \cd{camlp5}~\cite{Camlp5} and \cd{ppxlib}~\cite{PPXLib}), a runtime library and
a plugin system. The syntactic extensions process type definitions, attributed by an end user, and generate the following entities:

\begin{itemize}
\item a trasformation function (one per a type);
\item a virtual class, which is used as a common ancestor for all concrete transformations;
\item a number of concrete classes (one per requested plugin);
\item a \emph{typeinfo} structure, which incorporates the transformation function and a bundle of plugin-generated concrete functions, represented as an immediate object.
\end{itemize}

For example, for a type ``\lstinline{t}'' and requested plugins ``\lstinline{show}'', ``\lstinline{gmap}'' and ``\lstinline{fold}'' the structure with the following skeleton is
generated (``$\dots$'' stands for the parts we omit for now):

\begin{lstlisting}
   let $\inbr{transform_t}$ $\dots$ = $\dots$
   
   class virtual [$\dots$] $\inbr{t}$ $\dots$ =
   object
     $\dots$
   end

   class [$\dots$] $\inbr{show_t}$ $\dots$ =
   object inherit [$\dots$] $\inbr{t}$ $\dots$
     $\dots$
   end

   class [$\dots$] $\inbr{gmap_t}$ $\dots$ =
   object inherit [$\dots$] $\inbr{t}$ $\dots$
     $\dots$
   end

   class [$\dots$] $\inbr{fold_t}$ $\dots$ =
   object inherit [$\dots$] $\inbr{t}$ $\dots$
     $\dots$
   end

   let t = {
     gcata   = $\inbr{transform_t}$;
     plugins = object
                 method show = $\dots$
                 method gmap = $\dots$
                 method fold = $\dots$
               end
   }
\end{lstlisting}

Using the typeinfo structure we may mimick the type-indexed nature of the transformations:

\begin{lstlisting}
   let transform x = x.gcata
   let show      x = x.plugins#show
   let gmap      x = x.plugins#gmap
   let fold      x = x.plugins#fold
\end{lstlisting}

Thus, ``\lstinline{transform(t)}'' becomes the representation for ``$\inbr{transform_t}$'' in concrete syntax, etc. In the
Figure~\ref{syntax} we describe the concrete constructs, implemented by the syntax extensions.

\subsection{Types of Transformations}

The design of the library is based on the idea to describe transformations (e.g. catamorphisms~\cite{Bananas}) in terms of attribute grammars~\cite{AGKnuth,AGSwierstra,ObjectAlgebrasAttribute}.
In short, we consider only the transformations of the following type

\[
\iota \to t \to \sigma
\]

where $t$ is the type of a value to transform, $\iota$ and $\sigma$~--- types for \emph{inherited} and \emph{synthesized} values. We do not use attribute
grammars as a mean to describe the algorithmic part of transformations; we only utilize their terminology to describe the types of transformations. 

When the type under consideration is parameterized, the transformation becomes parameterized as well:

\begin{tabular}{cl}
  $(\iota_1 \to \alpha_1 \to \sigma_1) \to$ & \\
  $\dots$                                  & \\
  $(\iota_k \to \alpha_k \to \sigma_k) \to$ & $\iota \to (\alpha_1,\dots,\alpha_k)\;t \to \sigma$
\end{tabular}

In general the argument-transforming functions operate on inherited values of different types and return synthesized values of different types.

\subsection{Fixed Point Combinator and Memoization}

\subsection{Polymorphic Variants}

\subsection{Mutual Recursion}


\section{Examples}

In this section we demonstrate some examples, written with the aid of our library. In this examples we will use \cd{camlp5} syntax extension,
although \cd{ppxlib} plugin can be used equally.

First, we consider a simple type to represent arithmetic expressions:

\begin{lstlisting}
@type expr = Var   of string
           | Add   of expr * expr
           | Mul   of expr * expr
           | Div   of expr * expr
           | Const of int         with fmap
\end{lstlisting}

Here we requested a feature \cd{fmap}, which implements the conventional functor semantics. Since the type is not polymorphic, the function \cd{fmap(expr)}
just copies its argument. Although the copying can be considered useful on its own, this result a bit disappointing. However, with the aid of our framework we
actually can acquire a number of useful transformations, taking the copying as the starting point. For example, given a state \cd{st} we can substitute the
values of all variables in this state in an expression:
\marginnote{We omitted a method's argument for value being transformed. Is it intentional?}

\begin{lstlisting}
  let substitute st e = fix
    (fun f ->
       transform(expr)
         (object inherit [_] @expr[fmap] f
            method c_Var _ x = Const (st x)
          end)
          ()
    ) e
\end{lstlisting}

Indeed, all we need is to redefine the copy behavior for constructor \cd{Var}. In order to do this we inherit from the class \cd{fmap} for the type
\cd{expr} (denoted by \cd{@expr[gmap]} in the snippet), and rewrite the method \cd{c\_Var} (note the use of generic function \cd{transform(expr)} and
fix point combinator). As it can be seen from this example, we needed to implement only ``the interesting'' part of the transformation. All other
functionality (recursive propagation through the whole data structure) is handled by a framework-generated code.

For another example we consider an expression simplifier, which performs all possible calculations with constants and utilizes some
simple arithmetic equalities like $0*x=0$ or $0+x=x$:

\begin{lstlisting}
  class simplifier f =
  object 
    inherit [_] @expr[fmap] f
    method c_Div _ x y =
      match f x, f y with
      | Const x, Const y -> Const (x / y)
      | x      , Const 1 -> x
      | x      , y       -> Div (x, y)
    method c_Mul _ x y =
      match f x, f y with
      | Const x, Const y        -> Const (x * y)
      | Const 0, _ | _, Const 0 -> Const 0
      | Const 1, y              -> y
      | x, Const 1              -> x
      | x, y                    -> Mul (x, y)
    method c_Add _ x y =
      match f x, f y with
      | Const x, Const y -> Const (x + y)
      | Const 0, y       -> y
      | x, Const 0       -> x
      | x, y             -> Add (x, y)
  end
\end{lstlisting}

Since the interesting part is concentrated in the class definition, we omitted the top-level function, which looks exactly like the previous one,
since we are still dealing with the same feature \cd{fmap}. The class definition is much longer, than the previous one, but this is
inevitable~--- the interesting part is that long, indeed.

Note, the simplifier we implemented is strict~--- it evaluates both operands of a multiplication even if the first is equal 0. We can implement
a non-strict simplifier on top of the strict one:

\begin{lstlisting}
  class ns_simplifier f =
  object 
    inherit simplifier f 
    method c_Mul _ x y =
      match f x with
      | Const 0 -> Const 0
      | Const 1 -> f y
      | Const x -> (match f y with
                    | Const y -> Const (x * y)
                    | y       -> Mul   (Const x, y)
                    )
      | x       -> (match f y with
                    | Const 0 -> Const 0
                    | Const 1 -> x
                    | y       -> Mul (x, y)
                    )
  end
\end{lstlisting}

Again, this definition consists of only interesting part.

Finally, with substitutions an simplifications we can define an evaluation (first substitute, then simplify). Thus, the object layer of our framework
provides us with the powerful tool to create and modify transformations.

For another example we take the support for polymorphic variants~\cite{PolyVar,PolyVarReuse}, which we consider an important feature since it complements
the opportunity to provide composable data structures with the opportunity to create composable transformations. For the concrete problem we take the
transformation from named to nameless representations for lambda terms.

First, we define the generic part of the terms:

\begin{lstlisting}
  @type ('name, 'lam) lam = [
  | `App of 'lam * 'lam
  | `Var of 'name
  ] with eval
\end{lstlisting}

The \cd{eval} plugin here generates a transformation \cd{eval(lam)}, which is analogous to \cd{fmap}, but additionally uses some environment, which
by default is propagated unchanged. We here follow~\cite{PolyVarReuse} and use an open non-recursive definition of the type; our \cd{eval} corresponds
to \cd{map} in terms of~\cite{Visitors}.

Then, we define a binding construct~--- abstraction:

\begin{lstlisting}
  @type ('name, 'term) abs = [
  | `Abs of 'name * 'term
  ] with eval
  
  class ['term, 'term2] de_bruijn ft =
  object
    inherit [string, unit, 'term, 'term2,
             string list, 'term2] @abs[eval]
      (fun _ -> assert false)
      (fun _ _ -> ())
      ft
    method c_Abs env name term =
      `Abs ((), ft (name :: env) term)
  end
\end{lstlisting}

This time we have to define a conversion transformation since for the abstraction the default behavior of \cd{eval} is not
enough. We introduce the subclass for \cd{@abs[eval]}, in which we specify the type of the environment (\lstinline{string list}),
the representations for names in the input and output values (\lstinline{string} and \lstinline{unit} respectively), and
representations for subterms in the input and output values (abstract for now). The last, sixth type parameter for \cd{@abs[eval]}
is needed for open recursion. The semantics of the single method of this class reflects the normal behavior of the
abstraction during the conversion into the nameless representation: it adds the variable to the environment and uses this
environment to convert the subterm. The parameter \lstinline{ft} corresponds to the subterm conversion transformation. Since
we do not know it yet, we have to abstract over it.

Now we can combine two types into the single type for lambda terms:

\begin{lstlisting}
  @type ('n, 'b) term = [
  | ('n, ('n, 'b) term) lam
  | ('b, ('n, 'b) term) abs
  ] with eval

  @type named    = (string, string) term
  @type nameless = (int, unit) term
\end{lstlisting}

Here we distinguish names in binder positions (\lstinline{'b}) and bound positions (\lstinline{'n}) since their behavior during the
transformation essentially different: names in binder positions are erased, while in bound positions are substituted with corresponding
de Bruijn index. We also define shortcuts for the terms in named and nameless representations.

Similarly to the types, the transformations can be combined as well:

\begin{lstlisting}
  class de_bruijn fself =
  object
    inherit [string, int, string, unit,
             string list, nameless] @term[eval]
       fself
       ith
       (fun _ _ -> ())
    inherit [named, nameless] Abs.de_bruijn fself
  end
\end{lstlisting}

For the generic part of the terms we reused the \cd{eval} transformation, while for abstractions we took the customized one (\lstinline{de_bruijn}); in
any case the final transformation is build via inheritance with no other glue; here \lstinline{ith} is a function, which finds names in an
environment and returns their indices.

It is interesting, that with polymorphic variants is becomes possible to define a transformation with an output type, different from the input
beyond parameterization:

\begin{lstlisting}
   class ['term, 'term2] de_bruijn' ft =
   object
     inherit [string, string list, unit,
              'term, string list, 'term2,
              string list, 'term2, 'term] @abs
     method c_Abs env name term =
       `Abs (ft (name :: env) term) 
   end
     
   @type named = [
   | (string, named) lam
   | (string, named) abs
   ] with eval
                     
   @type nameless = [
   | (int, nameless) lam
   | `Abs of nameless
   ] with eval

   class de_bruijn fself =
   object
     inherit [string, int,
              named, nameless,
              string list,
              nameless] @lam[eval] fself ith fself
      inherit [named, nameless] Abs .de_bruijn' fself 
   end
\end{lstlisting}

Please note the implementation of method \lstinline{c_Abs}~--- now it returns a constructor \lstinline{`Add} with \emph{one}
argument. In short, we defined a transformation into a nameless representation, which completely removes the names in binder
positions.

\section{Related Works}

As our work makes use of both functional (combinators) and object-oriented (classes and objects) features of \textsc{OCaml} there are some relevant works
in both domains of typeful functional and object-oriented programming. The most relevant framework, developed for \textsc{OCaml}, which utilizes the same
ideas, but makes essentially different design decisions, is \textsc{Visitors}~\cite{Visitors}; we postpone the in-depth comparison of our framework with
\textsc{Visitors} until the end of this section.

First, there is a number of frameworks for generic programming in \textsc{OCaml}, which utilize a completely generative approach~\cite{Yallop,PPXDeriving}~---
all requested generic functions for all types are generated by the framework separately. This approach is very practical as long as the assortment
of shipped functions is rich enough and sufficient for a given use case. However, if not, someone has to extend the framework, implementing
all missing functions anew (and, potentially, with a very little code reuse). In addition, the functions themselves are hard coded and
lack extensibility. With our framework, first, many end-user generic functions can be easily derived from the generated ones, and second, in order to
implement a completely fresh plugin it is sufficient to hard code only ``the interesting'' part, as the generation of the single traversal
function and transformation class are already provided by the framework itself.

A number of approaches to functional generic programming utilizes the idea of type \emph{representation}~\cite{Hinze}.
The idea is to develop a uniform representation for any type under transformation and provide two conversion functions from- and to this representation
(ideally, comprising an isomorphism). A generic function performs transformation on a representation of actual data structure, which makes it possible to
implement every such function only once. The conversion functions themselves can in turn be constructed (semi) automatically using such features of
the language type system as type classes~\cite{Hinze,ALaCarte} or type families~\cite{InstantGenerics} (in \textsc{Haskell}) or generated using syntax extension
mechanism~\cite{GenericOCaml} (in \textsc{OCaml}). While some of these approaches allow extension and modification of generic functions by, for example, specifying a
specific treatment for some types or supporting extensible types, our solution is still more flexible as it allows modification with granularity of individual
constructors. In addition, with our framework it is possible for multiple versions of the same generic function for the same type to coexist.

A different approach is taken in ``Scrap Your Boilerplate'', or SYB~\cite{SYB}, initially developed for \textsc{Haskell}. This approach makes it
possible to implement transformations which identify the occurrencies of instances of a certain datatype inside arbitrary data structure. Two main
kinds of transformations are supported: \emph{queries}, which collect and return the instances of the designated datatype based on some user-defined
criterion, and \emph{transformations}, which uniformly propagate some type-preserving transformation for a datatype of interest. In the follow-up papers
the approach was extended to deal with transformations which traverse pairs of data structures~\cite{SYB1} and to support the extension of already implemented
transformations with new type cases~\cite{SYB2}. Later the approach was implemented  for other languages, including \textsc{OCaml}~\cite{SYBOCaml,Staged}.
Unlike our case, SYB takes the route of discriminating on a whole type, not individual constructors. In addition the shape of available transformations look rather
restrictive, and, once implemented, transformations for a given type can not be modified. It is interesting, that, potentially, SYB-style generic functions
can ``break through the ensapsulation barrier''~--- indeed, they can identify the occurrencies of values of type of interest inside \emph{arbitrarily typed}
data structures. Thus, their behaviour depend on the actual details of data structure organization, including those which were intentionally hidden by encapsulation.
This may result in, first, the possibility for undesirable reverse-engineering (by applying various type-sensitive transformations and analysing the results) and,
second, in fragility of interfaces~--- after a modification of data structure implementation generic functions for \emph{old} version can still be applied with
nither static nor dynamic error, but with wrong (or undesirable) results. 

There is a certain similarity between our approach and \emph{object algebras}~\cite{ObjectAlgebras}. Object algebras were proposed as a solution
for expression problem in mainstream object-oriented languages (\textsc{Java}, \textsc{C++}, \textsc{C\#}), which would not require advanced type system features besides
regular inheritance and generics. In the original exposition object algebras were presented as a design and implementation pattern; the follow-up
works have improved the initial proposal in various directions~\cite{ObjectAlgebrasAttribute,ObjectAlgebrasSYB}.
With object algebras a data structure under transformation is also encoded using the method-per-variant (constructor) idea, which makes it possible to
provide the extensibility in both dimensions and retroactive implementation. However, being developed for essentially different language environment,
the solution using object algebras would differ from ours in many concrete aspects. First, with object algebras the ``shape'' of a data structure has to
be represented by a generic function, which takes a concrete object algebra instance as a parameter (``Church encoding'' for types~\cite{Hinze}). Applying
this function to various implementations of object algebra one can acquire various transformations (for example, printing). To instantiate the data
structure itself one needs to provide a specific object algebra instance~--- \emph{factory}. However, after the instantiation the data structure itself
can not be generically transformed anymore. Thus, object algebras force end users to switch to data-as-function representation, which may or may not be
beneficial in different concrete cases. In contrast our approach non-destructively adds new functionality to the familiar world of algebraic data types,
pattern matching and recursive functions. Generic transformation implementations are completely separated from data representation, and end users may
freely transform their data structures in a familiar way without losing the ability to apply (or extend) generic functions. Another difference stems
from the fact that in our case, unlike mainstream object-oriented languages, polymorphic variants are used as a main tool for datatype extension.
Supporting polymorphic variants as a mean for datatype extensibility requires a fresh solution.



\section{Conclusion}
In this paper we presented an improved version of Generic Transformers, extended by support of PPX rewriters and type abbreviations. Although it 
uses the similar idea as in some related works, we claim that it allows to solve some problems in a more convenient manner.


\nocite{*}
\bibliographystyle{eptcs}
\bibliography{main}
\end{document}
