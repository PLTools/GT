\documentclass[submission,copyright,creativecommons]{eptcs}
\providecommand{\event}{ML 2018}  % Name of the event you are submitting to
\usepackage{breakurl}             % Not needed if you use pdflatex only.
\usepackage{underscore}           % Only needed if you use pdflatex.


\usepackage{booktabs} 
\usepackage{amssymb}
\usepackage{amsmath}
\usepackage{mathrsfs}
\usepackage{mathtools}
\usepackage{multirow}
\usepackage{listings}
\usepackage{indentfirst}
\usepackage{verbatim}
\usepackage{amsmath, amssymb}
\usepackage{graphicx}
\usepackage{xcolor}
\usepackage{url}
\usepackage{stmaryrd}
\usepackage{xspace}
\usepackage{comment}
\usepackage{wrapfig}
\usepackage[caption=false]{subfig}
\usepackage{placeins}
\usepackage{tabularx}
\usepackage{ragged2e}
\usepackage{soul}

\lstdefinelanguage{ocaml}{
keywords={@type, function, fun, let, in, match, with, when, class, type,
object, method, of, rec, repeat, until, while, not, do, done, as, val, inherit,
new, module, sig, deriving, datatype, struct, if, then, else, open, private, virtual, include, success, failure,
assert, true, false, end},
sensitive=true,
commentstyle=\small\itshape\ttfamily,
keywordstyle=\ttfamily\underbar,
identifierstyle=\ttfamily,
basewidth={0.5em,0.5em},
columns=fixed,
fontadjust=true,
literate={->}{{$\to$}}3 {===}{{$\equiv$}}1 {=/=}{{$\not\equiv$}}1 {|>}{{$\triangleright$}}3 {\\/}{{$\vee$}}2 {/\\}{{$\wedge$}}2 {^}{{$\uparrow$}}1,
morecomment=[s]{(*}{*)}
}

\lstset{
mathescape=true,
%basicstyle=\small,
identifierstyle=\ttfamily,
keywordstyle=\bfseries,
commentstyle=\scriptsize\rmfamily,
basewidth={0.5em,0.5em},
fontadjust=true,
language=ocaml
}

\newcommand{\cd}[1]{\texttt{#1}}

%\pagestyle{plain}
%\sloppy

\title{Generic Programming with Combinators and Objects}

\author{Dmitry Kosarev
  \institute{St. Petersburg State University\\
    JetBrains Research \\
    St. Petersburg, Russia}
\email{Dmitrii.Kosarev@protonmail.ch}
\and
Dmitry Boulytchev
\institute{St. Petersburg State University\\
  JetBrains Research \\
  St. Petersburg, Russia}
\email{dboulytchev@math.spbu.ru}
}

\def\titlerunning{Generic Programming with Combinators and Objects}
\def\authorrunning{D.Kosarev, D.Boulytchev}
\begin{document}
\maketitle

\begin{abstract}
  Abstract gies here
\end{abstract}

\section{Introduction}

Generic programming in the context of OCaml lately gains additional attention~\cite{Staged,Visitors,GenericOCaml}. 
We present a generic programming library GT\footnote{\url{https://github.com/kakadu/GT/tree/ppx-new}} (Generic Transformers), which has been in an active development and use since
2014. This library is an inheritor of our earlier work~\cite{SYBOCaml} on implementation of ``Scrap Your Boilerplate''
approach~\cite{SYB,SYB1,SYB2}. However, our experience has shown, that the extensibility of SYB is insufficient; in addition
the uniform transformations, based solely on type discrimination, turned out to be inconvenient to use (for example, they can allow
one to break through the encapsulation barrier). Our idea initially was to combine combinator and object-oriented approaches~--- the former
would provide means for parameterization, while the latter~--- for extensibility via late binding utilization. This idea in the form of
a certain design pattern was successfully evaluated~\cite{SCICO} and then reified in a library and a syntax extension~\cite{TransformationObjects}.
Our follow-up experience with the library~\cite{OCanren} has (once again) shown some flaws in the implementation. The version we present here is
almost a complete re-implementation with these flaws fixed.

From an end user perspective, our library is comprised of four layers:

\begin{enumerate}
\item On the top level it provides a syntax extension (in terms of both \cd{camlp5} and \cd{ppxlib}) with a number of plugins
(\cd{map}, \cd{fold}, \cd{show}, etc.) The interface of generated features is combinatorial, so their utilization is rather straightforward.
  
\item On the middle level it turns out, that all these features are implemented via object-encoded transformations with some reasonable
default behavior. This behavior can be modified/overridden using inheritance. Thus, customized transformations can be acquired using
the default ones.

\item On the low level it turns out, that all features are in fact instantiations of some very general transformation scheme; thus, transformations,
which do not fit in any pre-supplied plugin can still be implemented manually.
  
\item In the basement, the users can implement their own plugins; note, since all plugins are just instantiations of some generic scheme, the implementation
requires only a limited amount of work. In particular, all plugins use the same single traversal function, which need not to be generated.
\end{enumerate}

In fact, there is also an underground layer~--- all generic features are combined into an object, which can be passed as a parameter or modified. While
currently the library does not contain any conventional interface to deal with the object, it can be provided in the future (which opens a potentially
interesting opportunities for integration with existing proposals for \emph{ad-hoc} polymorphism~\cite{ModularImplicits}).

\section{Design}

The design of the library is based on the idea to describe transformations (e.g. catamorphisms~\cite{Bananas}) in terms of transformations, described by
attribute grammars~\cite{AGKnuth,AGSwierstra}. In short, we consider only the transformations of the following type

\[
\iota \to t \to \sigma
\]

where $t$ is the type of value to transform, $\iota$ and $\sigma$~--- types for \emph{inherited} and \emph{synthesized} values. We do not use attribute
grammars as a mean to describe the algorithmic part of transformations; we only utilize their terminology to describe the types of transformations.

When the type under consideration is parameterized, the transformation becomes parameterized as well:

\begin{tabular}{cl}
  $(\iota_1 \to \alpha_1 \to \sigma_1) \to$ & \\
  $\dots$                                  & \\
  $(\iota_k \to \alpha_k \to \sigma_k) \to$ & $\iota \to (\alpha_1,\dots,\alpha_k)\;t \to \sigma$
\end{tabular}

In general the argument-transforming functions operate on inherited values of different types and return synthesized values of different types.

The second idea is to encode a transformation for an algebraic data type as an object with per-constructor transformation methods (the similar idea is
used in~\cite{Visitors}). For example, for a type

\begin{lstlisting}
  type $\alpha$ t = A of $\alpha$ | B of $\alpha$ t * $\alpha$ t 
\end{lstlisting}

a transformation object would have the following structure

\begin{lstlisting}
  object
    method c_A : $\iota$ -> $\alpha$ -> $\sigma$
    method c_B : $\iota$ -> $\alpha$ t -> $\alpha$ t -> $\sigma$
  end
\end{lstlisting}

To automatically mass-produce transformation objects, a number of classes is generated: first, the common base virtual class for all transformations for
given type, and then one customized class per feature, requested by mean of plugins. All these classes are concrete, inherit from the
base one and are additionally parameterized by type parameters-transforming functions, including the function for transforming the type itself (thus
using open recursion pattern).

Finally, a single traversal function is generated. It takes a transformation object, an inherited attribute, and a value to traverse, performs pattern-matching
and calls appropriate methods of the object. For the example in question the traversal function may look like

\begin{lstlisting}
  let transform obj $\iota$ = function
  | A x      -> obj # c_A $\iota$ x
  | B (l, r) -> obj # c_B $\iota$ l r
\end{lstlisting}

Note, the traversal function is non-recursive; the recursion (if any) is indirectly handled in object's methods.

Within this infrastructure is turned out to be possible to implement such features as \cd{show}, \cd{fmap}, \cd{fold},
as well as \cd{eq} and \cd{compare}, which usually are expressed in an \emph{ad-hoc} manner in other frameworks. All these features are
implemented as plugins, which instantiate the generic components. Plugins also generate the top-level functions, tying the recursive knot,
and combine this functions into a data structure with the same name as the type of interest. All plugins supply a (universal) access function,
which takes this data structure as its first parameter. Under these conventions, \cd{show(int)} designates a \cd{show} function for \cd{int}s,
while \cd{fmap(list)}~--- \cd{fmap} for lists.

Beyond this simplified scheme some other things have to be done; for example, a special care has to be taken to support polymorphic variants, which
we consider an important feature of our library.

\section{Examples}

In this section we demonstrate some examples, written with the aid of our library. In this examples we will use \cd{camlp5} syntax extension,
although \cd{ppxlib} plugin can be used equally.

First, we consider a simple type to represent arithmetic expressions:

\begin{lstlisting}
  @type expr  = Var   of string
	      | Add   of expr * expr
	      | Mul   of expr * expr
	      | Div   of expr * expr
	      | Const of int with fmap
\end{lstlisting}

Here we requested a feature \cd{fmap}, which implements the conventional functor semantics. Since the type is not polymorphic, the function \cd{fmap(expr)}
just copies its argument. Although the copying can be considered useful on its own, this result a bit disappointing. However, with the aid of our framework we
actually can acquire a number of useful transformations, taking the copying as the starting point. For example, given a state \cd{st} we can substitute the
values of all variables in this state in an expression:

\begin{lstlisting}
  let substitute st e = fix
    (fun f ->
       transform(expr)
         (object inherit [_] @expr[fmap] f
            method c_Var _ x = Const (st x)
          end)
          ()
    ) e
\end{lstlisting}

Indeed, all we need is to redefine the copy behavior for constructor \cd{Var}. In order to do this we inherit from the class \cd{fmap} for the type
\cd{expr} (denoted by \cd{@expr[gmap]} in the snippet), and rewrite the method \cd{c\_Var} (note the use of generic function \cd{transform(expr)} and
fix point combinator). As it can be seen from this example, we needed to implement only ``the interesting'' part of the transformation. All other
functionality (recursive propagation through the whole data structure) is handled by a framework-generated code.

For another example we consider an expression simplifier, which performs all possible calculations with constants and utilizes some
simple arithmetic equalities like $0*x=0$ or $0+x=x$:

\begin{lstlisting}
  class simplifier f =
  object inherit [_] @expr[fmap] f
    method c_Div _ x y =
      match f x, f y with
      | Const x, Const y -> Const (x / y)
      | x      , Const 1 -> x
      | x      , y       -> Div (x, y)
    method c_Mul _ x y =
      match f x, f y with
      | Const x, Const y        -> Const (x * y)
      | Const 0, _ | _, Const 0 -> Const 0
      | Const 1, y              -> y
      | x, Const 1              -> x
      | x, y                    -> Mul (x, y)
    method c_Add _ x y =
      match f x, f y with
      | Const x, Const y -> Const (x + y)
      | Const 0, y       -> y
      | x, Const 0       -> x
      | x, y             -> Add (x, y)
  end
\end{lstlisting}

Since the interesting part is concentrated in the class definition, we omitted the top-level function, which looks exactly like the previous one,
since we are still dealing with the same feature \cd{fmap}. The class definition is much longer, than the previous one, but this is
inevitable~--- the interesting part is that long, indeed.

Note, the simplifier we implemented is strict~--- it evaluates both operands of a multiplication even if the first is equal 0. We can implement
a non-strict simplifier on top of the strict one:

\begin{lstlisting}
  class ns_simplifier f =
  object inherit simplifier f 
    method c_Mul _ x y =
      match f x with
      | Const 0 -> Const 0
      | Const 1 -> f y
      | Const x -> (match f y with                      
                    | Const y -> Const (x * y)
                    | y       -> Mul   (Const x, y)
                   )
      | x       -> (match f y with
                    | Const 0 -> Const 0
                    | Const 1 -> x
                    | y       -> Mul (x, y)
                   )
  end
\end{lstlisting}

Again, this definition consists of only interesting part.

Finally, with substitutions an simplifications we can define an evaluation (first substitute, then simplify). Thus, the object layer of our framework
provides us with the powerful tool to create and modify transformations.

For another example we take the support for polymorphic variants~\cite{PolyVar,PolyVarReuse}, which we consider an important feature since it complements
the opportunity to provide composable data structures with the opportunity to create composable transformations. For the concrete problem we take the
transformation from named to nameless representations for lambda terms.

First, we define the generic part of the terms:

\begin{lstlisting}
  @type ('name, 'lam) lam = [
  | `App of 'lam * 'lam
  | `Var of 'name
  ] with eval
\end{lstlisting}

The \cd{eval} plugin here generates a transformation \cd{eval(lam)}, which is analogous to \cd{fmap}, but additionally uses some environment, which
by default is propagated unchanged. We here follow~\cite{PolyVarReuse} and use an open non-recursive definition of the type; our \cd{eval} corresponds
to \cd{map} in terms of~\cite{Visitors}.

Then, we define a binding construct~--- abstraction:

\begin{lstlisting}
  @type ('name, 'term) abs = [
  | `Abs of 'name * 'term
  ] with eval
  
  class ['term, 'term2] de_bruijn ft =
  object
    inherit [string, unit, 'term, 'term2,
             string list, 'term2] @abs[eval]
      (fun _ -> assert false)
      (fun _ _ -> ())
      ft
    method c_Abs env name term =
      `Abs ((), ft (name :: env) term)
  end
\end{lstlisting}

This time we have to define a conversion transformation since for the abstraction the default behavior of \cd{eval} is not
enough. We introduce the subclass for \cd{@abs[eval]}, in which we specify the type of the environment (\lstinline{string list}),
the representations for names in the input and output values (\lstinline{string} and \lstinline{unit} respectively), and
representations for subterms in the input and output values (abstract for now). The last, sixth type parameter for \cd{@abs[eval]}
is needed for open recursion. The semantics of the single method of this class reflects the normal behavior of the
abstraction during the conversion into the nameless representation: it adds the variable to the environment and uses this
environment to convert the subterm. The parameter \lstinline{ft} corresponds to the subterm conversion transformation. Since
we do not know it yet, we have to abstract over it.

Now we can combine two types into the single type for lambda terms:

\begin{lstlisting}
  @type ('n, 'b) term = [
  | ('n, ('n, 'b) term) lam
  | ('b, ('n, 'b) term) abs
  ] with eval

  @type named    = (string, string) term
  @type nameless = (int, unit) term
\end{lstlisting}

Here we distinguish names in binder positions (\lstinline{'b}) and bound positions (\lstinline{'n}) since their behavior during the
transformation essentially different: names in binder positions are erased, while in bound positions are substituted with corresponding
DeBruijn index. We also define shortcuts for the terms in named and nameless representations.

Similarly to the types, the transformations can be combined as well:

\begin{lstlisting}
  class de_bruijn fself =
  object
    inherit [string, int, string, unit,
             string list, nameless] @term[eval]
       fself
       ith
       (fun _ _ -> ())
    inherit [named, nameless] Abs.de_bruijn fself
  end
\end{lstlisting}

For the generic part of the terms we reused the \cd{eval} transformation, while for abstractions we took the customized one (\lstinline{de_bruijn}); in
any case the final transformation is build via inheritance with no other glue; here \lstinline{ith} is a function, which finds names in an
environment and returns their indices.

It is interesting, that with polymorphic variants is becomes possible to define a transformation with an output type, different from the input
beyond parameterization:

\begin{lstlisting}
   class ['term, 'term2] de_bruijn' ft =
   object
     inherit [string, string list, unit,
              'term, string list, 'term2,
              string list, 'term2, 'term] @abs
     method c_Abs env name term =
       `Abs (ft (name :: env) term) 
   end
     
   @type named = [
   | (string, named) lam
   | (string, named) abs
   ] with eval
                     
   @type nameless = [
   | (int, nameless) lam
   | `Abs of nameless
   ] with eval

   class de_bruijn fself =
   object
     inherit [string, int,
              named, nameless,
              string list,
              nameless] @lam[eval] fself ith fself
      inherit [named, nameless] Abs .de_bruijn' fself 
   end
\end{lstlisting}

Please note the implementation of method \lstinline{c_Abs}~--- now it returns a constructor \lstinline{`Add} with \emph{one}
argument. In short, we defined a transformation into a nameless representation, which completely removes the names in binder
positions.

\section{Related Works}

As our work makes use of both functional (combinators) and object-oriented (classes and objects) features of \textsc{OCaml} there are some relevant works
in both domains of typeful functional and object-oriented programming. The most relevant framework, developed for \textsc{OCaml}, which utilizes the same
ideas, but makes essentially different design decisions, is \textsc{Visitors}~\cite{Visitors}; we postpone the in-depth comparison of our framework with
\textsc{Visitors} until the end of this section.

First, there is a number of frameworks for generic programming in \textsc{OCaml}, which utilize a completely generative approach~\cite{Yallop,PPXDeriving}~---
all requested generic functions for all types are generated by the framework separately. This approach is very practical as long as the assortment
of shipped functions is rich enough and sufficient for a given use case. However, if not, someone has to extend the framework, implementing
all missing functions anew (and, potentially, with a very little code reuse). In addition, the functions themselves are hard coded and
lack extensibility. With our framework, first, many end-user generic functions can be easily derived from the generated ones, and second, in order to
implement a completely fresh plugin it is sufficient to hard code only ``the interesting'' part, as the generation of the single traversal
function and transformation class are already provided by the framework itself.

A number of approaches to functional generic programming utilizes the idea of type \emph{representation}~\cite{Hinze}.
The idea is to develop a uniform representation for any type under transformation and provide two conversion functions from- and to this representation
(ideally, comprising an isomorphism). A generic function performs transformation on a representation of actual data structure, which makes it possible to
implement every such function only once. The conversion functions themselves can in turn be constructed (semi) automatically using such features of
the language type system as type classes~\cite{Hinze,ALaCarte} or type families~\cite{InstantGenerics} (in \textsc{Haskell}) or generated using syntax extension
mechanism~\cite{GenericOCaml} (in \textsc{OCaml}). While some of these approaches allow extension and modification of generic functions by, for example, specifying a
specific treatment for some types or supporting extensible types, our solution is still more flexible as it allows modification with granularity of individual
constructors. In addition, with our framework it is possible for multiple versions of the same generic function for the same type to coexist.

A different approach is taken in ``Scrap Your Boilerplate'', or SYB~\cite{SYB}, initially developed for \textsc{Haskell}. This approach makes it
possible to implement transformations which identify the occurrencies of instances of a certain datatype inside arbitrary data structure. Two main
kinds of transformations are supported: \emph{queries}, which collect and return the instances of the designated datatype based on some user-defined
criterion, and \emph{transformations}, which uniformly propagate some type-preserving transformation for a datatype of interest. In the follow-up papers
the approach was extended to deal with transformations which traverse pairs of data structures~\cite{SYB1} and to support the extension of already implemented
transformations with new type cases~\cite{SYB2}. Later the approach was implemented  for other languages, including \textsc{OCaml}~\cite{SYBOCaml,Staged}.
Unlike our case, SYB takes the route of discriminating on a whole type, not individual constructors. In addition the shape of available transformations look rather
restrictive, and, once implemented, transformations for a given type can not be modified. It is interesting, that, potentially, SYB-style generic functions
can ``break through the ensapsulation barrier''~--- indeed, they can identify the occurrencies of values of type of interest inside \emph{arbitrarily typed}
data structures. Thus, their behaviour depend on the actual details of data structure organization, including those which were intentionally hidden by encapsulation.
This may result in, first, the possibility for undesirable reverse-engineering (by applying various type-sensitive transformations and analysing the results) and,
second, in fragility of interfaces~--- after a modification of data structure implementation generic functions for \emph{old} version can still be applied with
nither static nor dynamic error, but with wrong (or undesirable) results. 

There is a certain similarity between our approach and \emph{object algebras}~\cite{ObjectAlgebras}. Object algebras were proposed as a solution
for expression problem in mainstream object-oriented languages (\textsc{Java}, \textsc{C++}, \textsc{C\#}), which would not require advanced type system features besides
regular inheritance and generics. In the original exposition object algebras were presented as a design and implementation pattern; the follow-up
works have improved the initial proposal in various directions~\cite{ObjectAlgebrasAttribute,ObjectAlgebrasSYB}.
With object algebras a data structure under transformation is also encoded using the method-per-variant (constructor) idea, which makes it possible to
provide the extensibility in both dimensions and retroactive implementation. However, being developed for essentially different language environment,
the solution using object algebras would differ from ours in many concrete aspects. First, with object algebras the ``shape'' of a data structure has to
be represented by a generic function, which takes a concrete object algebra instance as a parameter (``Church encoding'' for types~\cite{Hinze}). Applying
this function to various implementations of object algebra one can acquire various transformations (for example, printing). To instantiate the data
structure itself one needs to provide a specific object algebra instance~--- \emph{factory}. However, after the instantiation the data structure itself
can not be generically transformed anymore. Thus, object algebras force end users to switch to data-as-function representation, which may or may not be
beneficial in different concrete cases. In contrast our approach non-destructively adds new functionality to the familiar world of algebraic data types,
pattern matching and recursive functions. Generic transformation implementations are completely separated from data representation, and end users may
freely transform their data structures in a familiar way without losing the ability to apply (or extend) generic functions. Another difference stems
from the fact that in our case, unlike mainstream object-oriented languages, polymorphic variants are used as a main tool for datatype extension.
Supporting polymorphic variants as a mean for datatype extensibility requires a fresh solution.


\section{GT vs. Visitors}

Among existing generic programming frameworks for OCaml we can name two, which resemble ours: \cd{ppx\_deriving} and \cd{Visitors}~\cite{Visitors}.

\cd{Visitors}, on the other hand, explore a similar to ours object-oriented approach, in which many decisions, rejected by us, were taken (and vice versa). Here
we summarize the main differences:

\begin{itemize}
   \item \cd{Visitors} are excessively object-oriented~--- in order to use them one needs to instantiate some object and call a proper method. In our case as long as
     only predefined features are required a user can use a more native combinatorial interface.
     
   \item \cd{Visitors} implement a number of useful transformations in an \emph{ad-hoc} manner; in our case all transformations are instances of the
     same generic scheme. It is possible to combine different transformations via inheritance as long as the types of underlying scheme unify. We also argue, that
     in our framework the implementation of user-defined plugins is much easier.
     
   \item Following SYB, \cd{Visitors} take a type-discriminating route: for each type of interest (including the built-in ones) there is a dedicated
     transformation method in each object, representing a transformation. While this solution indeed adds some flexibility, we firmly oppose it, since it
     breaks the abstraction: inspecting the methods of a transformation (which cannot be hidden in a module signature) one can retrieve some
     information about the implementation of encapsulated types. Even worse, the data structures of abstract types can be manipulated in an unprescribed
     manner using the public type-transforming interface.

   \item In our case the type parameters for transformation classes have to be specified by an end user. With \cd{Visitors} this burden is offloaded to the
     compiler with the aid of some neat trick. However, this trick makes it impossible to use \cd{Visitors} syntax extension in module signatures. There is no
     such problem in our case~--- our framework can be equally used in both implementation and interface files.

   \item \cd{Visitors} in their current state do not support polymorphic variants.
\end{itemize}


\section{Conclusion}
In this paper we presented an improved version of Generic Transformers, extended by support of PPX rewriters and type abbreviations. Although it 
uses the similar idea as in some related works, we claim that it allows to solve some problems in a more convenient manner.

\begin{comment}
\section{Introduction}

The optional arguments of {\tt $\backslash$documentclass$\{$eptcs$\}$} are
\begin{itemize}
\item at most one of
{\tt adraft},
{\tt submission} or
{\tt preliminary},
\item at most one of {\tt publicdomain} or {\tt copyright},
\item and optionally {\tt creativecommons},
  \begin{itemize}
  \item possibly augmented with
    \begin{itemize}
    \item {\tt noderivs}
    \item or {\tt sharealike},
    \end{itemize}
  \item and possibly augmented with {\tt noncommercial}.
  \end{itemize}
\end{itemize}
We use {\tt adraft} rather than {\tt draft} so as not to confuse hyperref.
The style-file option {\tt submission} is for papers that are
submitted to {\tt $\backslash$event}, where the value of the latter is
to be filled in in line 2 of the tex-file. Use {\tt preliminary} only
for papers that are accepted but not yet published. The final version
of your paper that is to be uploaded at the EPTCS website should have
none of these style-file options.

By means of the style-file option
\href{http://creativecommons.org/licenses/}{creativecommons}
authors equip their paper with a Creative Commons license that allows
everyone to copy, distribute, display, and perform their copyrighted
work and derivative works based upon it, but only if they give credit
the way you request. By invoking the additional style-file option {\tt
noderivs} you let others copy, distribute, display, and perform only
verbatim copies of your work, but not derivative works based upon
it. Alternatively, the {\tt sharealike} option allows others to
distribute derivative works only under a license identical to the
license that governs your work. Finally, you can invoke the option
{\tt noncommercial} that let others copy, distribute, display, and
perform your work and derivative works based upon it for
noncommercial purposes only.

Authors' (multiple) affiliations and emails use the commands
{\tt $\backslash$institute} and {\tt $\backslash$email}.
Both are optional.
Authors should moreover supply
{\tt $\backslash$titlerunning} and {\tt $\backslash$authorrunning},
and in case the copyrightholders are not the authors also
{\tt $\backslash$copyrightholders}.
As illustrated above, heuristic solutions may be called for to share
affiliations. Authors may apply their own creativity here \cite{multipleauthors}.

Exactly 46 lines fit on a page.
The rest is like any normal {\LaTeX} article.
We will spare you the details.
The rest is like any normal {\LaTeX} article.
We will spare you the details.\\
The rest is like any normal {\LaTeX} article.
We will spare you the details.\\
The rest is like any normal {\LaTeX} article.
We will spare you the details.\\
The rest is like any normal {\LaTeX} article.
We will spare you the details.\\
The rest is like any normal {\LaTeX} article.
We will spare you the details.\hfill6\\
The rest is like any normal {\LaTeX} article.
We will spare you the details.\\
The rest is like any normal {\LaTeX} article.
We will spare you the details.\\
The rest is like any normal {\LaTeX} article.
We will spare you the details.\\
The rest is like any normal {\LaTeX} article.
We will spare you the details.\\
The rest is like any normal {\LaTeX} article.
We will spare you the details.\hfill11\\
The rest is like any normal {\LaTeX} article.
We will spare you the details.\\
The rest is like any normal {\LaTeX} article.
We will spare you the details.

Here starts a new paragraph. The rest is like any normal {\LaTeX} article.
We will spare you the details.
The rest is like any normal {\LaTeX} article.
We will spare you the details.\\
The rest is like any normal {\LaTeX} article.
We will spare you the details.\hfill16\\
The rest is like any normal {\LaTeX} article.
We will spare you the details.\\
The rest is like any normal {\LaTeX} article.
We will spare you the details.\\
The rest is like any normal {\LaTeX} article.
We will spare you the details.\\
The rest is like any normal {\LaTeX} article.
We will spare you the details.\\
The rest is like any normal {\LaTeX} article.
We will spare you the details.\hfill21\\
The rest is like any normal {\LaTeX} article.
We will spare you the details.\\
The rest is like any normal {\LaTeX} article.
We will spare you the details.\\
The rest is like any normal {\LaTeX} article.
We will spare you the details.\\
The rest is like any normal {\LaTeX} article.
We will spare you the details.\\
The rest is like any normal {\LaTeX} article.
We will spare you the details.\hfill26\\
The rest is like any normal {\LaTeX} article.
We will spare you the details.\\
The rest is like any normal {\LaTeX} article.
We will spare you the details.\\
The rest is like any normal {\LaTeX} article.
We will spare you the details.\\
The rest is like any normal {\LaTeX} article.
We will spare you the details.\\
The rest is like any normal {\LaTeX} article.
We will spare you the details.\hfill31\\
The rest is like any normal {\LaTeX} article.
We will spare you the details.\\
The rest is like any normal {\LaTeX} article.
We will spare you the details.\\
The rest is like any normal {\LaTeX} article.
We will spare you the details.\\
The rest is like any normal {\LaTeX} article.
We will spare you the details.\\
The rest is like any normal {\LaTeX} article.
We will spare you the details.\hfill36\\
The rest is like any normal {\LaTeX} article.
We will spare you the details.\\
The rest is like any normal {\LaTeX} article.
We will spare you the details.\\
The rest is like any normal {\LaTeX} article.
We will spare you the details.\\
The rest is like any normal {\LaTeX} article.
We will spare you the details.\\
The rest is like any normal {\LaTeX} article.
We will spare you the details.\hfill41\\
The rest is like any normal {\LaTeX} article.
We will spare you the details.\\
The rest is like any normal {\LaTeX} article.
We will spare you the details.\\
The rest is like any normal {\LaTeX} article.
We will spare you the details.\\
The rest is like any normal {\LaTeX} article.
We will spare you the details.\\
The rest is like any normal {\LaTeX} article.
We will spare you the details.\hfill46\\
The rest is like any normal {\LaTeX} article.
We will spare you the details.
The rest is like any normal {\LaTeX} article.
We will spare you the details.
The rest is like any normal {\LaTeX} article.
We will spare you the details.
The rest is like any normal {\LaTeX} article.
We will spare you the details.
The rest is like any normal {\LaTeX} article.
We will spare you the details.
The rest is like any normal {\LaTeX} article.
We will spare you the details.
The rest is like any normal {\LaTeX} article.
We will spare you the details.
The rest is like any normal {\LaTeX} article.
We will spare you the details.
The rest is like any normal {\LaTeX} article.
We will spare you the details.
The rest is like any normal {\LaTeX} article.
We will spare you the details.
The rest is like any normal {\LaTeX} article.
We will spare you the details.
The rest is like any normal {\LaTeX} article.
We will spare you the details.
The rest is like any normal {\LaTeX} article.
We will spare you the details.
The rest is like any normal {\LaTeX} article.
We will spare you the details.
The rest is like any normal {\LaTeX} article.
We will spare you the details.
The rest is like any normal {\LaTeX} article.
We will spare you the details.
The rest is like any normal {\LaTeX} article.
We will spare you the details.

\section{Ancillary files}

Authors may upload ancillary files to be linked alongside their paper.
These can for instance contain raw data for tables and plots in the
article, or program code.  Ancillary files are included with an EPTCS
submission by placing them in a directory \texttt{anc} next to the
main latex file. See also \url{https://arxiv.org/help/ancillary_files}.
Please add a file README in the directory \texttt{anc}, explaining the
nature of the ancillary files, as in
\url{http://eptcs.org/paper.cgi?226.21}.

\section{Prefaces}

Volume editors may create prefaces using this very template,
with {\tt $\backslash$title$\{$Preface$\}$} and {\tt $\backslash$author$\{\}$}.

\section{Bibliography}

We request that you use
\href{http://eptcs.web.cse.unsw.edu.au/eptcs.bst}
{\tt $\backslash$bibliographystyle$\{$eptcs$\}$}
\cite{bibliographystylewebpage}, or one of its variants
\href{http://eptcs.web.cse.unsw.edu.au/eptcsalpha.bst}{eptcsalpha},
\href{http://eptcs.web.cse.unsw.edu.au/eptcsini.bst}{eptcsini} or
\href{http://eptcs.web.cse.unsw.edu.au/eptcsalphaini.bst}{eptcsalphaini}
\cite{bibliographystylewebpage}. Compared to the original {\LaTeX}
{\tt $\backslash$biblio\-graphystyle$\{$plain$\}$},
it ignores the field {\tt month}, and uses the extra
bibtex fields {\tt eid}, {\tt doi}, {\tt ee} and {\tt url}.
The first is for electronic identifiers (typically the number $n$
indicating the $n^{\rm th}$ paper in an issue) of papers in electronic
journals that do not use page numbers. The other three are to refer,
with life links, to electronic incarnations of the paper.

\paragraph{DOIs}

Almost all publishers use digital object identifiers (DOIs) as a
persistent way to locate electronic publications. Prefixing the DOI of
any paper with {\tt http://dx.doi.org/} yields a URI that resolves to the
current location (URL) of the response page\footnote{Nowadays, papers
  that are published electronically tend
  to have a \emph{response page} that lists the title, authors and
  abstract of the paper, and links to the actual manifestations of
  the paper (e.g.\ as {\tt dvi}- or {\tt pdf}-file). Sometimes
  publishers charge money to access the paper itself, but the response
  page is always freely available.}
of that paper. When the location of the response page changes (for
instance through a merge of publishers), the DOI of the paper remains
the same and (through an update by the publisher) the corresponding
URI will then resolve to the new location. For that reason a reference
ought to contain the DOI of a paper, with a life link to the corresponding
URI, rather than a direct reference or link to the current URL of
publisher's response page. This is the r\^ole of the bibtex field {\tt doi}.
{\bf EPTCS requires the inclusion of a DOI in each cited paper, when available.}

DOIs of papers can often be found through
\url{http://www.crossref.org/guestquery};\footnote{For papers that will appear
  in EPTCS and use \href{http://eptcs.web.cse.unsw.edu.au/eptcs.bst}
  {\tt $\backslash$bibliographystyle$\{$eptcs$\}$} there is no need to
  find DOIs on this website, as EPTCS will look them up for you
  automatically upon submission of a first version of your paper;
  these DOIs can then be incorporated in the final version, together
  with the remaining DOIs that need to found at DBLP or publisher's webpages.}
the second method {\it Search on article title}, only using the {\bf
surname} of the first-listed author, works best.  
Other places to find DOIs are DBLP and the response pages for cited
papers (maintained by their publishers).

\paragraph{The bibtex fields {\tt ee} and {\tt url}}

Often an official publication is only available against payment, but
as a courtesy to readers that do not wish to pay, the authors also
make the paper available free of charge at a repository such as
\url{arXiv.org}. In such a case it is recommended to also refer and
link to the URL of the response page of the paper in such a
repository.  This can be done using the bibtex fields {\tt ee} or {\tt
url}, which are treated as synonyms.  These fields should \textbf{not} be used
to duplicate information that is already provided through the DOI of
the paper.
You can find archival-quality URL's for most recently published papers
in DBLP---they are in the bibtex-field {\tt ee}---but please suppress
repetition of DOI information. In fact, it is often
useful to check your references against DBLP records anyway, or just find
them there in the first place.

\paragraph{Typesetting DOIs and URLs}

When using {\LaTeX} rather than {\tt pdflatex} to typeset your paper, by
default no linebreaking within long URLs is allowed. This leads often
to very ugly output, that moreover is different from the output
generated when using {\tt pdflatex}. This problem is repaired when
invoking \href{http://eptcs.web.cse.unsw.edu.au/breakurl.sty}
{\tt $\backslash$usepackage$\{$breakurl$\}$}: it allows linebreaking
within links and yield the same output as obtained by default with
{\tt pdflatex}. 
When invoking {\tt pdflatex}, the package {\tt breakurl} is ignored.

Please avoid using {\tt $\backslash$usepackage$\{$doi$\}$}, or
{\tt $\backslash$newcommand$\{\backslash$doi$\}$}.
If you really need to redefine the command {\tt doi}
use {\tt $\backslash$providecommand$\{\backslash$doi$\}$}.

The package {\tt $\backslash$usepackage$\{$underscore$\}$} is
recommended to deal with underscores in DOIs. This is not needed when
using {\tt $\backslash$usepackage$\{$breakurl$\}$} and not {\tt pdflatex}.

\paragraph{References to papers in the same EPTCS volume}

To refer to another paper in the same volume as your own contribution,
use a bibtex entry with
\begin{center}
  {\tt series    = $\{\backslash$thisvolume$\{5\}\}$},
\end{center}
where 5 is the submission number of the paper you want to cite.
You may need to contact the author, volume editors or EPTCS staff to
find that submission number; it becomes known (and unchangeable)
as soon as the cited paper is first uploaded at EPTCS\@.
Furthermore, omit the fields {\tt publisher} and {\tt volume}.
Then in your main paper you put something like:

\noindent
{\small \tt $\backslash$providecommand$\{\backslash$thisvolume$\}$[1]$\{$this
  volume of EPTCS, Open Publishing Association$\}$}

\noindent
This acts as a placeholder macro-expansion until EPTCS software adds
something like

\noindent
{\small \tt $\backslash$newcommand$\{\backslash$thisvolume$\}$[1]%
  $\{\{\backslash$eptcs$\}$ 157$\backslash$opa, pp 45--56, doi:\dots$\}$},

\noindent
where the relevant numbers are pulled out of the database at publication time.
Here the newcommand wins from the providecommand, and {\tt \small $\backslash$eptcs}
resp.\ {\tt \small $\backslash$opa} expand to

\noindent
{\small \tt $\backslash$sl Electronic Proceedings in Theoretical Computer Science} \hfill and\\
{\small \tt , Open Publishing Association} \hfill .

\noindent
Hence putting {\small \tt $\backslash$def$\backslash$opa$\{\}$} in
your paper suppresses the addition of a publisher upon expansion of the citation by EPTCS\@.
An optional argument like
\begin{center}
  {\tt series    = $\{\backslash$thisvolume$\{5\}[$EPTCS$]\}$},
\end{center}
overwrites the value of {\tt \small $\backslash$eptcs}.
\end{comment}

\nocite{*}
\bibliographystyle{eptcs}
\bibliography{main}
\end{document}
